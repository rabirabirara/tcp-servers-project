\documentclass[letterpaper,twocolumn,10pt]{article}

% to be able to draw some self-contained figs
\usepackage{tikz}
\usepackage{amsmath}

% to include images
\usepackage{graphicx}

% inlined bib file
% \usepackage{filecontents}

% inline code listings
\usepackage{listings}
\usepackage{minted}

% Block quotes
\usepackage{csquotes}

% Define easy monospaced command.
\newcommand{\code}[1]{
    \mintinline{python}{#1}
}


%%% NOTES

%%%

\begin{document}

%don't want date printed
\date{}

% make title bold and 14 pt font (Latex default is non-bold, 16 pt)
\title{\Large \bf Using \code{asyncio} to facilitate an application server herd}

\maketitle


\begin{abstract}

    \code{asyncio} is a Python library which provides both high and low level APIs for writing
    asynchronous I/O code with the ubiquitous \code{async/await} syntax.  Its high-level APIs 
    cover common cases for single-threaded networking, while its low-level APIs allow for greater
    control over the coroutines being run.  Python's simplicity and popularity correlate
    to powerful standard library modules, and \code{asyncio} is no exception. Combined with
    Python's sweeping memory safety, single-threaded networking concurrency is easy.

\end{abstract}


\section{Python as a Language}

\subsection{Ease of Use}

Python is a notoriously simple language.  Compared to languages popular for concurrent networking such
as JavaScript, C\#, and even Go, Python's philosophy of clarity and brevity provides a smooth programming
experience (Python being an oft-used language for prototyping, case in point this project), as well as
a codebase well-suited for future analysis and augmentation.

Python's choice of syntax for coroutines, in \code{async/await}, is ubiquitous; it appears in the first
two languages mentioned above (JS and C\#), as well as in newer languages such as Rust, Kotlin, Dart,
and Nim.  While coroutines started in theory in the 1960s, the \code{async/await} syntax is only a
decade old - yet it has taken over mainstream concurrent programming.  Other languages with notable
facilities for concurrent programming, such as Go and Erlang, typically implement some form of
lightweight multithreading (see: 'goroutines' and 'Erlang processes'), as opposed to coroutines as 
seen in Python and JavaScript; in all languages with coroutines, some form of the 
\code{async/await} paradigm is used - even in C++20, which uses keywords like \code{co_await} and
\code{future<T>} to signify when functions are executed asynchronously.


\subsection{Compatibility}

\code{asyncio} is a relatively new library (having been added to the language in 2014) and as such
frequently experiences changes to its API.  The largest example of an update in API is the addition
of \code{asyncio.run()}, which occured in Python 3.7, and was made stable in Python 3.8.  The 
documentation for Python's 3.8 update itself acknowledges the breaking change - when comparing the new
function to old methods of starting the event loop, it remarks that:

\begin{displayquote}
    The actual implementation is significantly more complex. Thus, asyncio.run() should be the preferred way of running asyncio programs.
\end{displayquote}

While by now, the \code{asyncio} library is at least 7 years old, it is worth considering the problems of
future maintenance and updates to the codebase.  There are plenty of slight changes and deprecations to
parts of the library here and there; this may put engineers in consistent jobs as codebase janitors.


\subsection{Clarity and Semantics}

Python suffers from a problem with explicitness.  It is dynamically typed, so the code loses
not only an awareness of the types of arguments being serialized/de-serialized, encoded/decoded, but also
built-in documentation on the flow of execution through type annotation.  Nonetheless, the tradeoff for
loss of information in exchange for improvement in development agility should be considered when
discussing the lack of a type system.

This is also a problem of exception handling itself - a programmer cannot know what sort of exceptions a
function throws until they catch one themselves.  The documentation for the library is sparse when it
comes to describing all the possible ways things might go wrong.  When it comes to testing all the
possible ways the program might err or fail, Python may seem more hands on and empirical.  Many other
languages come with more detailed/complete error description elements; e.g. Rust and Haskell.

% Other languages are naturally more suited to good documentation and clarity than Python.  For example,
% Rust, which shares the \code{async/await} notation with Python, supplies the 'await' as a suffixed
% operator rather than a prefixed one - the value can then be chained into other expressions.  In Python,
% an await expression can only take one task.


\subsection{Memory-management, threading}

Memory-management should not be a problem in Python - the language is entirely memory-safe.  The same
cannot be said for some lower-level competitors (C++ and unsafe Rust), but it is a tradeoff of
performance for safety.  In typical memory safety, Python and Java are roughly equivalent - both languages
use garbage collection and use exception handling.  Python's null value is perhaps safer than Java's, 
and the lack of pointers/references makes Python less powerful but certainly safer.  It wouldn't be
inaccurate to claim that Python's cautious safety is superior to that of most competing languages -
the cost is performance and control.

Multithreading in Python is done not for performance, but for concurrent I/O.  The documentation for 
the standard library's \code{threading} states:

\begin{displayquote}
    In CPython, due to the Global Interpreter Lock, only one thread can execute Python code at once....
    threading is still an appropriate model if you want to run multiple I/O-bound 
    tasks simultaneously.
\end{displayquote}

The whole purpose of \code{asyncio} is to provide single-threaded concurrent I/O, and as such, no threads
are used or need to be used in the implementation of a web-server (the prototype certainly does not need
them).  This design reflects that of Node.js', the primary counterpart to \code{asyncio}.


\section{Working with \code{asyncio}}

\subsection{API}

The \code{asyncio} high-level API is exceedingly simple and high-level.  There are two main functions:
\code{open_connection} is for establishing a network connection as a client, and \code{start_server}
is for setting up a socket server.  These two functions provide all the functionality needed for most
cases, and certainly for the app server prototype we designed.

There is also of course a low-level API for programmers who need it, which should raise confidence that
the library carries enough facilities to implement a large web app stack.

Being a new API, there are enough oddities that might hinder the speed of development under 
\code{asyncio}.  For example, if you want to run a set of coroutines concurrently, the documentation
suggests creating \code{Task}s out of them.  To run multiple \code{Task}s, the documentation suggests
using \code{asyncio.gather()}.  However, the function does not take an iterable, but a variadic 
sequence of arguments - so, if you want to run $n$ many of these tasks concurrently, with $n$ variable 
during runtime (in an iterable), you have to know what the 'splat' operator is - a mild inconvenience
for what should be a common use case of passing a series of coroutines to a function designed to
run multiple coroutines at the same time.

Nonetheless, \code{asyncio} is easy to write and maintain - its syntax resembles synchronous Python,
with only the addition of two keywords to acknowledge the event loop.  One possible drawback is that
everything in the \code{asyncio} environment must also be made asynchronous, but with such a simple
API, extending programs with coroutines galore should not be problematic.

\subsection{Latest features}

For \code{asyncio}, it is absolutely preferable to write code using the newest features of Python.  If
the code is being written from scratch, obviously the latest (if not pre-release) stable API should be
used, to help defend against slow deprecation.  If we had an old codebase already written, the
\code{asyncio} API does support the same operations it promoted as first-class options in earlier
versions of the library, with minor changes, so that it shouldn't be too problematic to use older
versions.  The server prototype in fact uses very few elements from the most recent version of the
library - nonetheless, \code{asyncio.run()} may prove very convenient when writing a server without
needing to understand the implemntation underneath.


\section{Discussing Performance}

\subsection{Python is not the fastest}

Python is undeniably a slow language.  If speed is a powerful concern, using compiled languages like
Go, C\#, or Rust are all better options.  For example, if the data sent back and forth 
requires a lot of internal processing, the servers might be caught up in CPU-bound operations 
rather than I/O-bound operations, which defeats the purpose of using the library 
(i.e. that I/O is the primary bottleneck); Node.js suffers from the same problem.  
There are C-based extensions to the language, as well
as JIT compilers (PyPy comes to mind) - however, these are not practical to extend and are often
incompatible with recent versions of \code{asyncio}.  Regardless, it isn't expected that language 
performance is a huge issue for the application discussed.


\subsection{\code{asyncio} and co. are not the fastest}

It would be hard to describe \code{asyncio} itself as the fastest interface for asynchronous networking
available to Python, and especially not in comparison with other languages' libraries.  Near the time
of its introduction, \code{aiohttp} notably was bottlenecked by a slow HTTP parser - which
eventually was replaced with a significantly faster parser written in C.  For performance improvements,
extensions related to C are common.


\subsubsection*{\code{uvloop}}

There also exists the library \code{uvloop}, which replaces the \code{asyncio} built-in event loop with
a reputedly faster event loop.  Its benchmarks boast TCP server performance comparable to Golang's,
almost twice that of Node.js's performance. While its benchmarks are dated by now (2016), it may be worth
considering - especially as it implements the same interface as the \code{asyncio} event loop, allowing
for a seamless transition between the two implementations.  In effect, we would be using \code{asyncio},
but with a free performance bonus.  One caveat is that the \code{uvloop} library only works on Unix
machines - but this shouldn't be a problem as servers generally run distributions of Linux.


\subsection{Synchronous, multithreaded?}

The design choice of stackful coroutines in a single-threaded event loop over a multithreaded one is
ultimately better.  If we are to use Python, multithreaded performance suffers, thanks to the Global
Interpreter Lock.  In other languages which better support threading for performance, there is still
a massive consumption of resources for having to expend a new thread's stack for each request.

Asynchronous programming is less wasteful and, when written properly, can provide a tighter flow
of execution, without costing as much memory/OS involvement.


\section{Conclusion}

The change in architecture with \code{asyncio} as the driving force is a good decision.  The library
provides effortless asynchronous networking capabilities, in all of conception, prototyping, development,
and maintenance - this is due to Python's inherent simplicity and brevity.  While some extensions or
adjustments to the environment may be desirable for the sake of performance, ultimately the \code{asyncio}
library is a fine choice for our application.


\end{document}


